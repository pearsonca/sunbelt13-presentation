%%This is a template for creating a slide presentation
%%using the seminar package (be sure to install the 
%%seminar package to build this.)

\documentclass{seminar}
\usepackage{amsmath}
%\newcommand{\mytoday}{\day \month \year}

\begin{document}

\begin{slide}
\title{Hello World}
Carl A. B. Pearson \\
Postdoctoral Researcher \\
Emerging Pathogens Institute, University of Florida \\
last compiled: \today
\end{slide}

\begin{slide}
Topic: Seasonal Vector Populations \\
comparison of continuous, spatially homogenous models % deliberately simple basis

Why? 

understanding of infection trends can inform interventions, health system preparations, {\em etc}.
\end{slide}

\begin{slide}
Common Mosquito Model % in disease contexts, not used absolutely everywhere

TODO sine
\end{slide}

\begin{slide}
...{\em vs.} Common Mosquito Abundance

TODO overlay mosquito pops with sine, match peaks
\end{slide}

\begin{slide}
So: low hanging fruit

preview: no sophisticated analysis to pick said fruit \\
but these basic analyses provide fertile ground for much more quantitative detail % note about the value of this approach in general to science engineering
\end{slide}

\begin{slide}
\title{Engineering \&\ Maths Refresher, I}
\begin{enumerate}
\item useful to write models in terms of measurable parameters,
\item measurable parameters are not scale-free,
\item mathematics is more useful when scale free, therefore
\item dimensional analysis is awesome
\end{enumerate}
\end{slide}

\begin{slide}
where $M(t)$ is mosquito population w.r.t time
\begin{align*}
\dot{M(t)} = E(t) - \lambda M(t)
\end{align*}
defined on $t\in(-T/2,T/2)$
\end{slide}

\begin{slide}
common usage is $M(t)\propto$ simple trigonometric

What salient observed features does that miss?

aside: why replace given $M(t)$ with given $\dot{M(t)}$?
\end{slide}

\begin{slide}
Salient features:
\begin{itemize}
\item short time with appreciable population
% note: measure of pop. may be totally unreliable; however, trappable pop might good surrogate for active pop.
\item even shorter time for population rise and fall
\item low correlation with early and peak populations % accurate? provide comparison plots?
\end{itemize}

Need a spike-like $E(t)$ to replicate these.  Candidates?
\end{slide}

\begin{slide}
Spike-like could be more formally $\delta$-function like.

So: use $\delta$-function approximations.
\end{slide}

\begin{slide}
TODO list approximate delta functions.
\end{slide}

\begin{slide}
What should we use for the shape parameters?

clue: want oranges-to-oranges comparisons between the options
\end{slide}

\begin{slide}
I chose to make mosquito total births equivalent

TODO $M_p$ equation

and then to apply a subjective ``constraint'' on $\Delta t$

TODO delta t stuff
\end{slide}

\begin{slide}
TODO list approximate delta functions with params in place
\end{slide}

\begin{slide}
Now everything is on the same scale, but rewind: don't have any of the convenience of having the same scale.

So: dimensional analysis.  What parameters should be eliminated?
\end{slide}

\end{document}
